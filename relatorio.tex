\documentclass[10pt,a4paper]{article}

\usepackage[table]{xcolor}
\usepackage[top=20mm, bottom=30mm, left=18mm, right=18mm]{geometry}
\usepackage{multirow}
\usepackage{amsmath}
\usepackage{hyperref}  % Enables hyperlinks

% Encoding
%--------------------------------------
\usepackage[utf8]{inputenc}
\usepackage[T1]{fontenc}
%--------------------------------------
%Portuguese-specific commands
%--------------------------------------
\usepackage[portuguese]{babel}
%--------------------------------------

\definecolor{light_gray}{RGB}{235, 235, 235}

\usepackage{graphicx}
\graphicspath{{./img/}}

\newcommand{\horrule}[1]{\rule{\linewidth}{#1}}

%----------------------------------------------------------------------------------------
%	TITLE SECTION
%----------------------------------------------------------------------------------------

\title{
	\usefont{OT1}{bch}{b}{n}
	\normalfont \Large \textsc{Universidade Federal de Santa Catarina\\
		Departamento de Informática e Estatística\\
		Inteligência Artificial
	} \\ [25pt]
	\horrule{0.5pt} \\[0.4cm]
	\huge Trabalho sobre Representação de Conhecimento e Raciocínio \\
	\horrule{2pt} \\[0.5cm]
}

\author{Leonardo V. Eichstaedt\\
        \and
        Makhles Reuter Lange
}
\date{\today}

\begin{document}
\maketitle

%----------------------------------------------------------------------------------------
%	PESQUISA TEÓRICA
%----------------------------------------------------------------------------------------

\section{Pesquisa teórica}

\subsection{\texttt{SubclassOf} vs \texttt{equivalentTo}}

Utiliza-se o axioma de subclasse \texttt{SubClassOf( CE$_1$ CE$_2$ )}, no qual \texttt{CE$_1$} e \texttt{CE$_2$} são expressões de classe, para construir o conceito de hierarquia de classes. Ou seja, \texttt{CE$_1$} é mais específica que \texttt{CE$_2$}.

O axioma de classes equivalentes \texttt{EquivalentClasses( CE$_1 \ldots$ CE$_n$ )}, por sua vez, estabelece que todas as expressões de classes \texttt{CE$_i$}, $1 \leq i \leq n$, são equivalentes entre si.

Dessa forma, utilizando o exemplo dado, tem-se que \texttt{Judoca subClassOf fazGolpe some Golpe} implica em qualquer instância de \texttt{Judoca} também possuir a propriedade \texttt{fazGolpe some Golpe}. No entanto, uma instância de \texttt{fazGolpe some Golpe} não é inferido como sendo também uma instância de \texttt{Judoca}, enquanto que ao se usar o axioma \texttt{equivalentTo}, o motor de inferência irá considerar ambas as instâncias equivalentes. Exemplo:

\begin{itemize}

    \item \texttt{MilitaryUnit subClassOf Unit} implica em uma instância de \texttt{MilitaryUnit} herdar as propriedades de \texttt{Unit}.

    \item \texttt{MilitaryUnit equivalentTo (attacks some (Unit or Building) and wields some Weapon)} implica em cada um dos indivíduos resultantes de \texttt{attacks some (Unit or Building) and wields some Weapon} também ser uma instância de \texttt{MilitaryUnit}.

\end{itemize}

%----------------------------------------------------------------------------------------

\subsection{HermiT}

Motores de inferência recebem coleções de axiomas e oferecem um conjunto de operações
nestes axiomas da ontologia. O primeiro passos realizado por um motor de inferência é realizar os testes de consistência e de satisfazibilidade. Caso satisfeitos, o motor irá tentar construir as inferências baseadas nas regras definidas. Essas inferências são feitas com base no modelo de implementação de inferência da ferramenta.

O HermiT é um motor de inferência para ontologias escritas em OWL. Dado um arquivo OWL, o HermiT pode determinar se uma ontologia é consistente, derivar inferências e mais. Ele baseia-se em \emph{hypertableau} calculus\footnote{Um tipo de cálculo completo para lógica de primeira ordem baseada em cláusulas.}, uma abordagem que permite lidar com problemas de
não-determinismo e tamanho do modelo. O HermiT utiliza uma otimização que tenta reusar
indivíduos existentes ao invés de gerar novos, além de contar com otimizações relacionadas a classificação de ontologias.

Como o HermiT é um \emph{tableau based reasoner}, ele tenta construir modelos da ontologia utilizando regras de realização nos indivíduos do modelos para garantir que todos eles e suas relações satisfaçam os axiomas da ontologia. Portanto, no caso do HermiT, por ser baseado em \emph{hypertableau}, ele reduz consideralvelmente o tamanho dos modelos construídos e as advinhações usadas para explorar as possíveis construções.

Um dos problemas para as regras de realização contruírem modelos que satisfaçam as
ontologias é ficar construindo modelos infinitamente até satisfazer os axiomas. Para resolver esse problema, utiliza-se um mecanismo de detecção de ciclos chamado \emph{blocking}.


%----------------------------------------------------------------------------------------
%	MODELO CONCEITUAL
%----------------------------------------------------------------------------------------

\section{Modelo Conceitual - Age of Empires 2}

O domínio escolhido para a criação da ontologia foi o jogo \emph{Age of Empires 2}. É um jogo de estratégia em tempo real que possui os modos de jogo \emph{single player}, através de campanhas, e \emph{multiplayer}. Neste último, o objetivo principal é a conquista dos reinos adversários. As principais características do jogo são:
%
\begin{itemize}
    \item O jogo é divido em idades: Idade Negra, Idade Feudal, Idade dos Castelos e Idade Imperial. Os jogadores iniciam a partida na Idade Negra.
    \item Os jogadores devem escolher uma dentre as diversas civilizações disponíveis, \emph{e.g.}, Bretões, Turcos, Chineses, Espanhóis, Hunos, etc.
    \item Cada civilização possui uma gama de construções, as quais podem ser do tipo militar (Quartel, Castelo, Ferraria, etc) ou relacionadas à economia (Centro da cidade, Mercado, Universidade, etc).
    \item Existem diversas unidades, que podem ser do tipo militar (são feitas em construções do tipo militar) ou do tipo relacionado à economia (por exemplo um camponês, um lenhador ou um carro de comércio).
    \item As unidades militares utilizam armas de curto (Espadas) ou longo alcance (Arco e Flechas, Lanças, Arma de Fogo).
    \item Para que um jogador avance para a próxima idade, é necessário que tenha feito determinadas construções e que doe uma certa quantidade de recursos (ouro, comida, madeira ou pedra) obtidos do mapa, que varia com a idade desejada.
    \item Algumas construções e unidades só estão disponíveis a partir de determinadas idades, \emph{e.g.}, Castelos e Monastérios podem ser construídos somente a partir da Idade dos Castelos.

\end{itemize}


%----------------------------------------------------------------------------------------
%   ONTOLOGIA
%----------------------------------------------------------------------------------------

\section{Ontologia - AoE2}

A ontologia foi criada utilizando-se o programa Protégé 5.0.0 e utilizando-se o motor de inferência HermiT 1.3.8.

\subsection{Classes}

A seguinte hierarquia de classes foi definida para a ontologia AoE2:
%
\begin{itemize}
    \item Age
        \begin{itemize}
            \item DarkAge
            \item FeudalAge
            \item CastleAge
            \item ImperialAge
        \end{itemize}
    \item Building
        \begin{itemize}
            \item EconomicBuilding
            \item MilitaryBuilding
                \begin{itemize}
                    \item Castle
                \end{itemize}
        \end{itemize}
    \item Civilization
    \item Mount
    \item Player
    \item Resource
    \item Unit
        \begin{itemize}
            \item EconomicUnit
            \item MilitaryUnit
                \begin{itemize}
                    \item MeleeUnit
                    \item RangedUnit
                \end{itemize}
        \end{itemize}
    \item Weapon
        \begin{itemize}
            \item MeleeWeapon
            \item RangedWeapon
        \end{itemize}
\end{itemize}
% 

%----------------------------------------------------------------------------------------
\subsection{Indivíduos}

Indivíduos simples, sem nenhum tipo de inferência realizado (destaca-se a classe a qual pertence o indivíduo):

\begin{itemize}
    \item \textbf{Resource} Food, Gold, Stone, Wood
    \item \textbf{Mount} Horse
    \item \textbf{Age} DarkAge, FeudalAge, CastleAge, ImperialAge
\end{itemize}

Indivíduos em que foi feita alguma inferência (OPA = Object Property Assertions; DPA = Data Property Assertions):

\begin{itemize}
    \item \textbf{Bow} DPA: distance 10; Tipo inferido: Weapon
    \item \textbf{Shield} DPA: distance 0; Tipo inferido: Weapon
    \item \textbf{Sword} DPA: distance 1; Tipo inferido: Weapon
    \item \textbf{Mill} OPA: storesResources Food; Tipo inferido: EconomicBuilding
    \item \textbf{LumberMill} OPA: storesResources Wood; Tipo inferido: EconomicBuilding
    \item \textbf{TownCenter} OPA: storesResources Food, storesResources Gold, storesResources Stone, storesResources Wood; Tipo inferido: EconomicBuilding
    \item \textbf{Lumberjack} OPA: gathersWood Wood; Tipo inferido: EconomicUnit; OPA inferido: gathersResources Wood
    \item \textbf{Fisherman} OPA: gathersFood Food; Tipo inferido: EconomicUnit; OPA inferido: gathersResources Food
    \item \textbf{Miner} OPA: gathersGold Gold, gathersStone Stone; Tipo inferido: EconomicUnit; OPA inferido: gathersResources Gold, gathersResources Stone
    \item \textbf{Villager} OPA: gathersResources Food, builds Mill, builds LumberMill, builds TownCenter; Tipo inferido: EconomicUnit
    \item \textbf{WindsorCastle} OPA: availableAt CastleAge, defendsFrom Knight; OPA inferido: isOwnedBy EdwardLongshanks; Tipo inferido: Castle
    \item \textbf{Knight} OPA: rides Horse, wields Shield, wields Sword; OPA inferidos: isOwnedBy EdwardLongshanks, attacks Mangudai; Tipo inferido: MilitaryUnit
    \item \textbf{Mangudai} OPA: attacks Knight, rides Horse, wields Bow; OPA inferido: isOwnedBy GenghisKhan; Tipo inferido: MilitaryUnit
    \item \textbf{EdwardLongshanks} OPA: owns WindsorCastle, owns Knight, hasEnemy GenghisKhan, is Britons; Tipo inferido: Player
    \item \textbf{GenghisKhan} OPA: owns Mangudai, is Mongols; OPA inferido: hasEnemy EdwardLongshanks; Tipo inferido: Player
\end{itemize}
% \begin{tabular}{ |l|c|c|c|c| }
%     Indivíduo & APO$^1$ & Tipo & APO$^1$ Inferido & Tipo Inferido \\
%     \hline
%     \multirow{4}{*}{Edward Longshanks} & owns WindsorCastle \\

%     \hline
% \end{tabular}

%----------------------------------------------------------------------------------------
\subsection{Propriedades e Restrições Aplicadas}


\subsubsection{Equivalent To}
Com relação às propriedades de dados, criou-se somente a propriedade \emph{distance}, cujo Domínio é a classe \texttt{Weapon} e a Imagem \texttt{xsd:integer}. O intuito dessa propriedade é ser utilizada nas definições de equivalência de \texttt{MeleeWeapon} e \texttt{RangedWeapon}:
%
\begin{description}
    \item [MeleeWeapon] \texttt{Equivalent To: Weapon and (distance max 1 owl:rational)}
    \item [RangedWeapon] \texttt{Equivalent To: Weapon and (distance min 3 owl:rational)}
    \item [EconomicUnit] \texttt{gathersResources some Resource}
    \item [MilitaryUnit] \texttt{attacks some (Building or Unit) and (wields some Weapon)}
    \item [MeleeUnit] \texttt{MilitaryUnit and (wields some MeleeWeapon)}
    \item [RangedUnit] \texttt{MilitaryUnit and (wields some RangedWeapon)}
    \item [Castle] \texttt{MilitaryBuilding and (defendsFrom some Unit)}
    \item [EconomicBuilding] \texttt{Building and (storesResources some Resource)}
\end{description}

\subsubsection{Disjoint With}
\begin{itemize}
    \item EconomigUnit e MilitaryUnit
    \item MeleeUnit e RangedUnit
    \item MeleeWeapon e RangedWeapon
\end{itemize}

\subsubsection{Different Individuals}

\begin{itemize}
    \item Food, Gold, Stone e Wood
\end{itemize}

%----------------------------------------------------------------------------------------
%	REFERÊNCIAS BIBLIOGRÁFICAS
%----------------------------------------------------------------------------------------

\section{Referências Bibliográficas}

HORROCKS, Ian, MOTIK, Boris, WANG, Zhe. The HermiT OWL Reasoner. Disponível em: \url{https://www.cs.ox.ac.uk/boris.motik/pubs/hmw12HermiT.pdf}. Acesso em: 10/10/2016.

\vspace{5mm}

SHEARER, Rob, MOTIK, Boris, HORROCKS, Ian. HermiT: A Highly-Efficient OWL Reasoner. Disponível em: \url{https://www.cs.ox.ac.uk/boris.motik/pubs/smh08HermiT.pdf}. Acesso em: 10/10/2016.

\vspace{5mm}

ONTOGENESIS, How does a reasoner work? Disponível em: \url{http://ontogenesis.knowledgeblog.org/1486}. Acesso em: 10/10/2016.

\vspace{5mm}

W3C, OWL 2 Web Ontology Language. Disponível em: \url{https://www.w3.org/TR/owl2-syntax/}. Acesso em: 8/10/2016.

\vspace{5mm}

LUGER, George. Inteligência Artificial. Estruturas e Estratégias para a solução de problemas complexos. 4ª edição. Bookman, 2004.


\end{document}
