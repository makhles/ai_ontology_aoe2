\documentclass[10pt,a4paper]{article}

\usepackage[table]{xcolor}
\usepackage[top=20mm, bottom=30mm, left=18mm, right=18mm]{geometry}
\usepackage{multirow}
\usepackage{amsmath}
\usepackage{hyperref}  % Enables hyperlinks

% Encoding
%--------------------------------------
\usepackage[utf8]{inputenc}
\usepackage[T1]{fontenc}
%--------------------------------------
%Portuguese-specific commands
%--------------------------------------
\usepackage[portuguese]{babel}
%--------------------------------------

\definecolor{light_gray}{RGB}{235, 235, 235}

\usepackage{graphicx}
\graphicspath{{./img/}}

\newcommand{\horrule}[1]{\rule{\linewidth}{#1}}

%----------------------------------------------------------------------------------------
%	TITLE SECTION
%----------------------------------------------------------------------------------------

\title{
	\usefont{OT1}{bch}{b}{n}
	\normalfont \Large \textsc{Universidade Federal de Santa Catarina\\
		Departamento de Informática e Estatística\\
		Inteligência Artificial
	} \\ [25pt]
	\horrule{0.5pt} \\[0.4cm]
	\huge Trabalho sobre Representação de Conhecimento e Raciocínio \\
	\horrule{2pt} \\[0.5cm]
}

\author{Leonardo V. Eichstaedt\\
        \and
        Makhles Reuter Lange
}
\date{\today}

\begin{document}
\maketitle

%----------------------------------------------------------------------------------------
%	PESQUISA TEÓRICA
%----------------------------------------------------------------------------------------

\section{Pesquisa teórica}

\subsection{\texttt{SubclassOf} vs \texttt{equivalentTo}}

Utiliza-se o axioma de subclasse \texttt{SubClassOf( CE$_1$ CE$_2$ )}, no qual \texttt{CE$_1$} e \texttt{CE$_2$} são expressões de classe, para construir o conceito de hierarquia de classes. Ou seja, \texttt{CE$_1$} é mais específica que \texttt{CE$_2$}.

O axioma de classes equivalentes \texttt{EquivalentClasses( CE$_1 \ldots$ CE$_n$ )}, por sua vez, estabelece que todas as expressões de classes \texttt{CE$_i$}, $1 \leq i \leq n$, são equivalentes entre si.

Dessa forma, utilizando o exemplo dado, tem-se que \texttt{Judoca subClassOf fazGolpe some Golpe} implica em qualquer instância de \texttt{Judoca} também possuir a propriedade \texttt{fazGolpe some Golpe}. No entanto, uma instância de \texttt{fazGolpe some Golpe} não é inferido como sendo também uma instância de \texttt{Judoca}, enquanto que ao se usar o axioma \texttt{equivalentTo}, o motor de inferência irá considerar ambas as instâncias equivalentes. Exemplo:

\begin{itemize}

    \item \texttt{MilitaryUnit subClassOf Unit} implica em uma instância de \texttt{MilitaryUnit} herdar as propriedades de \texttt{Unit}.

    \item \texttt{MilitaryUnit equivalentTo (attacks some (Unit or Building) and wields some Weapon)} implica em cada um dos indivíduos resultantes de \texttt{attacks some (Unit or Building) and wields some Weapon} também ser uma instância de \texttt{MilitaryUnit}.

\end{itemize}


%----------------------------------------------------------------------------------------
%	MODELO CONCEITUAL
%----------------------------------------------------------------------------------------

\section{Modelo Conceitual - Age of Empires 2}

O domínio escolhido para a criação da ontologia foi o jogo \emph{Age of Empires 2}. É um jogo de estratégia em tempo real que possui os modos de jogo \emph{single player}, através de campanhas, e \emph{multiplayer}. Neste último, o objetivo principal é a conquista dos reinos adversários. As principais características do jogo são:
%
\begin{itemize}
    \item O jogo é divido em idades: Idade Negra, Idade Feudal, Idade dos Castelos e Idade Imperial. Os jogadores iniciam a partida na Idade Negra.
    \item Os jogadores devem escolher uma dentre as diversas civilizações disponíveis, \emph{e.g.}, Bretões, Turcos, Chineses, Espanhóis, Hunos, etc.
    \item Cada civilização possui uma gama de construções, as quais podem ser do tipo militar (Quartel, Castelo, Ferraria, etc) ou relacionadas à economia (Centro da cidade, Mercado, Universidade, etc).
    \item Existem diversas unidades, que podem ser do tipo militar (são feitas em construções do tipo militar) ou do tipo relacionado à economia (por exemplo um camponês, um lenhador ou um carro de comércio).
    \item As unidades militares utilizam armas de curto (Espadas) ou longo alcance (Arco e Flechas, Lanças, Arma de Fogo).
    \item Para que um jogador avance para a próxima idade, é necessário que tenha feito determinadas construções e que doe uma certa quantidade de recursos (ouro, comida, madeira ou pedra) obtidos do mapa, que varia com a idade desejada.
    \item Algumas construções e unidades só estão disponíveis a partir de determinadas idades, \emph{e.g.}, Castelos e Monastérios podem ser construídos somente a partir da Idade dos Castelos.

\end{itemize}


%----------------------------------------------------------------------------------------
%   ONTOLOGIA
%----------------------------------------------------------------------------------------

\section{Ontologia - AoE2}

\subsection{Classes}

A seguinte hierarquia de classes foi definida para a ontologia AoE2:
%
\begin{itemize}
    \item Age
        \begin{itemize}
            \item DarkAge
            \item FeudalAge
            \item CastleAge
            \item ImperialAge
        \end{itemize}
    \item Attack
    \item Building
        \begin{itemize}
            \item EconomicBuilding
            \item MilitaryBuilding
        \end{itemize}
    \item Civilization
        \begin{itemize}
            \item Britons
            \item Chinese
            \item Huns
        \end{itemize}
    \item Mount
    \item Player
    \item Resource
    \item Unit
        \begin{itemize}
            \item EconomicUnit
            \item MilitaryUnit
        \end{itemize}
    \item Weapon
        \begin{itemize}
            \item MeleeWeapon
            \item RangedWeapon
        \end{itemize}
\end{itemize}
% 
%----------------------------------------------------------------------------------------
\subsection{Restrições Aplicadas}

%----------------------------------------------------------------------------------------
\subsection{Indivíduos}

\begin{tabular}{ l c c }
    Indivíduo & APO* & APD** \\
    \hline
    Player1 & owns Castle & playerName = "Edward Longshanks" \\
    \hline
\end{tabular}


\subsection{Inferências Realizadas}

%----------------------------------------------------------------------------------------
%	REFERÊNCIAS BIBLIOGRÁFICAS
%----------------------------------------------------------------------------------------

\section{Referências Bibliográficas}

W3C, OWL 2 Web Ontology Language. Disponível em: \url{https://www.w3.org/TR/owl2-syntax/}. Acesso em: 8/10/2016.

LUGER, George. Inteligência Artificial. Estruturas e Estratégias para a solução de problemas complexos. 4ª edição. Bookman, 2004.


\end{document}
